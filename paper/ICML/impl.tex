%!TEX root=paper.tex

\section{ShrinkNets in Practice}

In this section we discuss practical aspects of \shrink{}, including 
neuron removal  and several optimizations.

\subsection{On-The-Fly Neuron Removal}
\label{neuron_killing}

Switch layers are initialized with weights sampled from $\mathcal{N}(0,1)$;
their values change as part of the training process so as to switch \emph{on}
or \emph{off} neurons. When neurons are deactivated by switch layers, we need
to remove them to actually shrink the network size.
In the following we discuss three potential strategies to remove neurons. 
%Since after removing a neuron it will no longer be able to contribute to  prediction accuracy, we must devise strategies that remove neurons while remaining robust to changing gradient values. We observe three strategies:

\textbf{Threshold strategy}: Under this strategy neurons are removed based on
a fixed threshold expressed in terms of its absolute value. This is the method
used by deep compression \cite{Han2015}. We found this method is very sensitive to the initializaiton 
and not robust to noise in the gradients---which occurs commonly when training networks with
stochastic gradient descent---because the threshold depends on the scale of
each weight.

\textbf{Sign change strategy}: Under this strategy neurons are removed when the
weight value changes its sign. As before this strategy is also sensitive to gradient noise; if we sample the gradient in a way that is not fully representative of the dataset we might experience one-time zero crossing which could wrongly remove a neuron.

\textbf{Sign variance strategy}: Instead of removing on the first zero
crossing, we can instead measure the exponential moving variance of the sign
($-1, 1$) of each parameter in the \textit{switch layer}. When the value of
the exponential moving variances goes over a predefined threshold, we consider
the neuron deactivated, and therefore we remove it. This strategy introduces
two extra parameters, one to control the behavior of the inertia of the
variance, and the threshold. 
However, we found that this strategy performs the best in practice and is used throughout our evaluation section. 

%The last two strategies perform better than the threshold strategy.  Both effectively allow us to shrink the network by varying $\lambda$, and result in networks that are both small and perform well. We use the third one in our evaluation section. 

\subsection{Additional Optimizations for ShrinkNets}

\noindent\textbf{Preparing for Inference: } With ShrinkNets we obtain
reduced-sized networks during training, which is the first steps towards faster
inference. This networks are readily available for inference. However, because
they include switch layers---and therefore more parameters---they introduce
unnecessary overhead at inference time. To avoid this overhead, we reduce the
network parameters by combining each switch layer with its respective network
layer by multiplying the respective parameters before emitting the final trained
network. As a result, the final network is a dense network without any switching layers. 

\noindent\textbf{Neural Garbage Collection: }ShrinkNets decides on-the-fly which
neurons to remove. Since ShrinkNets removes a large fraction of neurons, we must
dynamically resize the network at runtime to not unnecessarily impact network
training time. We implemented a neural garbage collection method as part of our
library which takes care of updating the necessary network layers as well as
updating optimizer state to reflect the neuron removal.



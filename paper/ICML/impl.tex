%!TEX root=paper.tex

\section{ShrinkNets in Practice}

In this section we discuss practical aspects of ShrinkNets, including the
neuron removal  process and several additional optimizations.

%The previous section described the switch layers, which deactivate neurons
%on-the-fly. To benefit from ShrinkNets in practice, we must overcome two
%challenges. First, to effectively reduce the network size we must remove the
%deactivated neurons. Since removed neurons will no longer contribute to
%improving prediction accuracy, we must be confident when removing them. Second,
%for inference, we must remove the switch layers from the network. We describe
%our solution for both challenges in this section.

\subsection{On-The-Fly Neuron Removal}
\label{neuron_killing}

Switch layers are initialized with weights sampled from $\mathcal{N}(0,1)$;
their values change as part of the training process so as to switch \emph{on}
or \emph{off} neurons. When neurons are deactivated by switch layers, we need
to remove them to actually shrink the network size. Since after removing a
neuron it will no longer be able to contribute to  prediction accuracy, we
must devise strategies that remove neurons while remaining robust to changing
values. We observe three strategies:


\textbf{Threshold strategy}: Under this strategy neurons are removed based on
a fixed threshold expressed in terms of its absolute value. This is the method
used by deep compression \cite{}. We found this method is not robust to noise
in the gradients---which occurs commonly when training networks with
stochastic gradient descent---because the threshold depends on the scale of
each weight.

\textbf{Sign change strategy}: Under this strategy neurons are removed when the
weight value changes its sign. This strategy works well in practice but it is
also sensitive to gradient noise. If we sample the gradient in a way that is
not fully representative of the dataset we might  experience one-time zero
crossing which could wrongly kill a neuron.

\textbf{Sign variance strategy}: Instead of killing  on the first zero
crossing, we can instead measure the exponential moving variance of the sign
($-1, 1$) of each parameter in the \textit{switch layer}. When the value of
the exponential moving variances goes over a predefined threshold, we consider
the neuron deactivated, and therefore we remove it. This strategy introduces
two extra parameters, one to control the behavior of the inertia of the
variance, and the threshold, but we have found it is the best performing in
practice.

The last two strategies perform better than the threshold strategy.  Both
effectively allow us to shrink the network by varying $\lambda$, and result in
networks that are both small and perform well.

%According to our experiments on linear and logistic regressions the
%two last strategies reduce the number of parameter update performed during
%training by at least an order of magnitude with only minimal impact on the final
%error obtained by the model.

\subsection{Additional Optimizations for ShrinkNets}

\noindent\textbf{Preparing for Inference: } With ShrinkNets we obtain reduced-
sized networks during training, which is the first steps towards faster
inference. This networks are readily available for inference. However, because
they include switch layers---and therefore more parameters---they introduce
unnecessary overhead at inference time. To avoid this overhead, we reduce the
network parameters by combining each switch layer with its respective network
layer by multiplying the respective parameters before emitting the final
trained network. In the unlikely case that a switch layer is sandwiched
between two non-linearly scalable layers, we leave the switch layer as is.

\noindent\textbf{Neural Garbage Collection: }ShrinkNets decides on-the-fly
which neurons to remove. Since ShrinkNets removes a large fraction of neurons,
we must dynamically resize the network at runtime to not unnecessarily impact
network training time. We implemented a neural garbage collection method as
part of our library which takes care of updating the necessary network layers
as well as updating optimizer state to reflect the neuron removal.


%\ra{move to eval:implementation, retrieve the neuran garbace collection and
%place it somewhere here}
%We have implemented \textbf{Switch Layers} and the associated training procedure
%as a library in pytorch~\cite{paszke2017automatic}.  The layer can be freely
%mixed with other popular layers such as convolutional layers, batchnorm layers,
%fully connected layers etc.  There are two key challenges that must be overcome
%while implementing \textbf{Switch Layers}.  
%First, since ShrinkNets kill off a
%large fraction of neurons, we must remove the overhead of carrying around dead
%neurons.  And second, we must update the various network layers and training
%machinery (e.g.  optimizer state) to reflect the deactivation of neurons.  We
%address these challenges by augmenting every layer with additional information
%about neuron status and implementing a \emph{neural garbage collection}
%mechanism which prunes deactivated neurons on-the-fly and updates the state of
%the training machinery.

%A second implementation challenge we address with ShrinkNets is the potentially
%higher inference time.  This increase in inference time is both due to the extra
%computation introduced by switch layers and because our switch layers are not
%currently optimized for CUDA.  Therefore, for now, our library provides a
%post-training routine to ``fold'' switch layers into existing, optimized layers.
%Specifically, for every switch layer, we check if the parent or child of this
%layer is a linearly scalable layer (e.g. convolutional layer, fully connected
%layer, batchnormalization layer) and if so, we fold the switch layer into this
%neighboring layer by multiplying through by the switch weights.  
%In the unlikely
%case that a switch layer is sandwiched between two non-linearly scalable layers,
%we leave the switch layer as is.


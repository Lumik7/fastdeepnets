\section{Discussion}

\gl{Did not have time to write it but added some pointers}

\begin{itemize}
  \item Where does the method shine:  We shine where there are too many layers
    To explore the network size and we have to rely on heuristics. It shines by
    it simplicity compared to other methods \gl{Do we want to talk about the
      number of lines of codes ?}. Prunning while we train, from the compression
    results seem to be benificial to prune while training to achieve very good
    compression.
  \item Side-effects of method: smaller networks, time to train. There is not
    significant difference in training time, mostly because the filter layer is
    not optimized. However we can say the convergence speed is improved. For
    example for \texttt{CIFAR10} the mean number of epoch to reach the best is $62.5$ for \textit{ShrinkNets} vs $68.4$ for normal networks. for \texttt{COVERTYPE}
    the numbers are respectively $117.9$ and $130$.
  \item Potential extensions/limitations: We could add a filter that learn the
    size of convolutions for convolution layers. Supporting Residual Network
    would allow us to support more architecture. We could also try to mix try to
    learn the number of layers such is described in~\cite{meier}
\end{itemize}
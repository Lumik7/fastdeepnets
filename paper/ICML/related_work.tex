%!TEX root=paper.tex
\section{Related Work}

%\begin{itemize}
%  \item post-training compression techniques -- brain damage , 
%  \item group sparsity e.g., \cite{Scardapane2017} and non-parametric neural networks -- 
%  \item training dynamics paper: first overfitting and then randomization?, \gl{Here is the ref, if you can introduce it in the flow \cite{Shwartz-Ziv2017}}
%\end{itemize}

There are several lines of work related to optimizing network structure. 

%Given the importance of network structure, many researchers have explored the
%problem of finding the best network structure for a given learning task.  The
%proposed techniques broadly fall into five categories: random search andbrute
%force search, hyperparameter optimization, model compression after training,
%resizing models during training, and automated architecture search methods.

\noindent\textbf{Hyperparameter optimization techniques: }
One way to optimize network architecture is to use 
hyperparameter optimization.  Although many methods have been 
proposed, e.g., \cite{BergstraJAMESBERGSTRA2012,Snoek12},
randomized search has been shown to work surprising well.
%  Brute force search of network sizes is
%also become more practical due to faster and more powerful
%hardware~\cite{molchanov2016pruning}.  
The more complex methods for
hyperparameter optimization include techniques, e.g., ~\cite{Snoek12} typically
 select hyperparameter combinations that come from uncertain areas of the
hyperparameter space to search efficiency. 
As a generalization of this,  methods based on bandit algorithms (e.g.
~\cite{li2016hyperband, jamieson2016}) have also become a popular way to tune
hyperparameters by quickly discarding 
model configurations that perform badly. 
Although these methods could in theory be used to tune the number of neurons per layer
of a network, in practice no related work proposes this, because treating each layer as a hyperparameter
would lead an excessively large search space.
In contrast, with ShrinkNets the size of the network can be tuned with 
a single parameter, as we explain in the next section..
% More importantly, none of the hyperparameter optimization methods focuses on
% finding small networks, which is a crucial property of ShrinkNets, necessary to
% achieve good inference times.

%As noted before, all of the above techniques require many tens
%to hundreds of models to be trained, making this process computationally
%inefficient and slow.  More practically, the hyperparameter optimization
%literature does not evaluate their methods on network size and instead focuses
%on optimization hyperparameters such as learning rates and weight decay
%parameters.

\noindent\textbf{Model Compression: }Model compression techniques focus on
reducing the model size \emph{after} training, in contrast to ShrinkNets, which
reduces it \emph{while} training. 
Optimal brain damage~\cite{Cun} identifies connections in a network that are
unimportant and then prunes these connections.
DeepCompression~\cite{han2015deepcompression} takes this one step further and in
addition to pruning connections, it quantizes weights to make inference
extremely efficient.  A different vein of work such as ~\cite{romero2014fitnets,
hinton2015distilling} proposes techniques for distilling a network into a
simpler network or a different model. Because these techniques work after
training, they are orthogonal and complementary to ShrinkNets. Further,
some of these techniques, e.g.,~\cite{Han2015,Cun}, produce sparse matrices that
are not likely to improve inference times even though they reduce network size.
%Unlike our technique which works during
%training, these techiques are used after training and it would be interesting to
%apply them to ShrinkNets as well. 
%\cite{Abadi2016b} share the common goal of
%removing entire blocks of parameter to maintain dense matrices, however their
%method only applies to convolutional layers.

\noindent\textbf{Auto-ML: } Some work that focuses on automatically learning
model architecture through the use of genetic algorithms and reinforcement
learning techniques~\cite{DBLP:journals/corr/ZophL16, zoph2017learning}. These
techniques are focused on learning higher-level architectures (e.g.,
building blocks for neural network architectures). In particular, they do not
focus on finding small but well-performing networks for inference, which is the
goal of ShrinkNets.
\tim{Argument is not really convincing, but those techniques require to train full models and might take weeks to converge. }

\noindent\textbf{Dynamically Sizing Networks: }The techniques closest to our
proposed method are those based on group sparsity such as
~\cite{Scardapane2017}, and those like~\cite{Philipp} that dynamically grow and shrink
 the size of the network during training.  \cite{Scardapane2017}
presents a method that also deactivates neurons using a loss function based on
group-sparsity.  However, the exact details of how their method works are not
given, and their experimental results (on a small, fully connected network), are
substantially worse than ours as shown in Section~\ref{sec:experiments}.
\cite{Philipp} propose a method called Adaptive Radial-Angular Gradient Descent
that adds neurons on the fly and removes neurons via an $l_2$ penalty.  However,
this approach requires a new optimizer and takes longer to converge compared to
ShrinkNets.




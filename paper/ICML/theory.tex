%\subsection{Relation to Group Sparsity}
\noindent\textbf{Relation to Group Sparsity (LASSO)}

% \begin{equation}
%   \Omega_\lambda^s = \lambda \norm{\bm{\beta}}_1
% \end{equation}

Our goal when designing ShrinkNets was to be able to remove inputs
and outputs of layers. For classic fully connected layers, which are defined
as :

\begin{equation} \label{fully_connected}
  f_{\bm{A}, \bm{b}}(\bm{x})=a(\bm{Ax + b})
\end{equation}

removing an input neuron $j$ is equivalent to have $\left(\bm{A}^T\right)_j = \bm{0}$
and removing an output neuron $i$ is the same as having $\bm{A}_i = \bm{0}$ and $\bm{b}_i = 0$. Solving  optimization problems while trying to set entire groups
of parameters to zero has been already studied and the most popular method
is without doubt the group sparsity regularization [ref]. For any partitionning of the set of parameter defining a model in $p$ groups: $\bm{\theta} = \bigcup_{i=1}^P \bm{\theta}_i$ we define it the following way:

\begin{equation}
  \Omega_\lambda^{gp} = \lambda \sum_{i=1}^p \sqrt{\#\bm{\theta_i}} \norm{\bm{\theta_i}}_2 \\
\end{equation}

In the context of a fully-connected layer, the groups are either: columns of $\bm{A}$ if we want to remove inputs, or rows of $\bm{A}$ and the corresponding entry in $\bm{b}$ if we want to remove outputs. For simplicity, we will focus our analysis in the simple one-layer case. In this case filtering outputs does not make a lot of sense, this is why we will only consider the former case. The group sparsity regularization then becomes:


\begin{equation} \label{group_sparsity_regularization}
  \Omega_\lambda^{gp} = \lambda \sum_{j=1}^p \norm{\bm{\left(A^T\right)_j}}_2 \\
\end{equation}

Because $\forall i, \#\bm{\theta}_i = n$, To make the notation simpler,
we embedded $\sqrt{n}$ inside $\lambda$.

Since group sparsity and ShrinkNets try to achieve the same goal we will try to understand their similarities and differences. First let's recall the two problems. The original ShrinkNet problem is:

\begin{equation}
  \min_{\bm{A}, \bm{\beta}} \norm{\bm{y} - \bm{A}\diag{\bm{\beta}}\bm{x}}_2^2 + \lambda \norm{\bm{\beta}}_1
\end{equation}

And the Group Sparsity problem is:

\begin{equation}
  \min_{\bm{A}} \norm{\bm{y} - \bm{A}\bm{x}}_2^2 + \Omega_\lambda^{gp}
\end{equation}

We can prove the under the condition: $\forall j\in \intint{1, p}, \norm{\left(\bm{A}^T\right)_j}_2 = 1$ the two problems are equivalent (\cref{gps_equivalence}). However if we relax this constraint then shrinknet becomes non-convex and has no global minimum (\cref{unconstrained_non_convex,unconstrained_shrinknet_no_min}). Fortunately, by adding an extra term to the ShrinkNet regularization term we can proove that:

\begin{equation}
  \min_{\bm{A}, \bm{\beta}} \norm{\bm{y} - \bm{A}\diag{\bm{\beta}}\bm{x}}_2^2 + \Omega_\lambda^s + \lambda_2\norm{A}_p^p
\end{equation}

has many global minimum (\cref{shrinknet_regularized_minimum}) for all $p>0$. This is the reason we defined the \textit{regularized ShrinkNet penalty} earlier this way:

\begin{equation}
  \Omega_{\lambda, \lambda_2, p}^{rs} = \lambda\norm{\bm{\beta}}_1 + \lambda_2\norm{\bm\theta}_p^p
\end{equation}

In practice we observed that $p=2$ or $p=1$ are good choice, while the latter will also introduce additional sparsity in the parameters.

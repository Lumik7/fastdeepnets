%!TEX root=paper.tex
%\subsection{Relation to Group Sparsity}
\noindent\textbf{Relation to Group Sparsity (LASSO): } \shrink removes neurons,
i.e., inputs and outputs of layers. For a fully connected layer defined as:
%
\begin{equation} \label{fully_connected}
  f_{\bm{A}, \bm{b}}(\bm{x})=a(\bm{Ax + b})
\end{equation}
%
where $\bm{A}$ represents the connections and $\bm{b}$ the bias,
removing an input neuron $j$ is equivalent to having $\left(\bm{A}^T\right)_j =
\bm{0}$. Removing an output neuron $i$ is the same as setting $\bm{A}_i = \bm{0}$
and $\bm{b}_i = 0$. Solving optimization problems while trying to set entire
group of parameters to zero is the goal of group sparsity regularization
\cite{Scardapane2017}. 
In  any partitioning of the set of parameters $\bm{\theta}$ defining a model in $p$
groups: $\bm{\theta} = \bigcup_{i=1}^P \bm{\theta}_i$, group sparsity 
penalty is defined as (where $\lambda$ is the regularization parameter): 
%
\begin{equation}
  \Omega_\lambda^{gp} = \lambda \sum_{i=1}^p \sqrt{\mathbf{card}(\bm{\theta_i}}) \norm{\bm{\theta_i}}_2 \\
\end{equation}
% \srm{define notation -- what is $\lambda$;  what is $\#\theta$, what is $\Omega$? Where does the square root come from?} \gl{This is how it is defined in the
% original paper, it would require a few sentences to explain that, do really
% need to do ?}
In fully-connected layers, the groups are either: columns of
$\bm{A}$ if we want to remove inputs, or rows of $\bm{A}$ and the corresponding
entry in $\bm{b}$ if we want to remove outputs. For simplicity, we focus
our analysis in the simple one-layer case. In this case, filtering outputs does
not make sense, so we only consider removing inputs. The
group sparsity regularization then becomes (when $\sqrt{n}$ is folded into the $\lambda$)
%
\begin{equation} \label{group_sparsity_regularization}
  \Omega_\lambda^{gp} = \lambda \sum_{j=1}^p \norm{\bm{\left(A^T\right)_j}}_2 \\
\end{equation}

% Because $\forall i, \#\bm{\theta}_i = n$, embedded $\sqrt{n}$ inside $\lambda$.
%  \ra{is \# the general way of expressing
% cardinality? why not $|x|$?,}  \gl{In europe |x| is for the abolute value, and I was using it in the notation section that someone removed, therefore there would have been a notation conflict}to make the notation simpler, we

Group sparsity and \shrink try to achieve the same goal. We discuss next how
they are related to each other. First let's recall the two problems. In the context of approximating $\bm{y}$ with a linear regression from features $\bm{x}$), the
original \shrink problem is 
%
\begin{equation}
  \min_{\bm{A}, \bm{\beta}} \norm{\bm{y} - \bm{A}\diag{\bm{\beta}}\bm{x}}_2^2 + \lambda \norm{\bm{\beta}}_1
\end{equation}
%
And the group sparsity problem is:
%
\begin{equation}
  \min_{\bm{A}} \norm{\bm{y} - \bm{A}\bm{x}}_2^2 + \Omega_\lambda^{gp}
\end{equation}
%

We can prove that under the condition: $\forall j\in \intint{1, p},
\norm{\left(\bm{A}^T\right)_j}_2 = 1$ the two problems are equivalent
(\cref{gps_equivalence}, see Apendix). However if we relax this constraint then \shrink
becomes non-convex and has no global minimum
(\cref{unconstrained_non_convex,unconstrained_shrinknet_no_min}, also in Appendix). Fortunately,
by adding an extra term to the \shrink regularization term we can prove that:
%
\begin{equation}
  \min_{\bm{A}, \bm{\beta}} \norm{\bm{y} - \bm{A}\diag{\bm{\beta}}\bm{x}}_2^2 + \Omega_\lambda^s + \lambda_2\norm{A}_p^p
\end{equation}
%
has global minimums for all $p>0$ (\cref{shrinknet_regularized_minimum}).
This is the reason we defined the regularized \shrink penalty above as:
%
\begin{equation}
  \Omega_{\lambda, \lambda_2, p}^{rs} = \lambda\norm{\bm{\beta}}_1 + \lambda_2\norm{\bm\theta}_p^p
\end{equation}
%
In practice we observed that $p=2$ or $p=1$ are good a choice; note that the latter
will also introduce additional sparsity into the parameters because the $L_1$ is, 
thest best convex approximation of the $L_0$ norm.
%!TEX root=paper.tex

\section{The ShrinkNets Approach}

In this section describe the
ShrinkNets approach, which allows us to learn the network size as part of the
training process. The key idea is the addition of 
\emph{switch layers} added after each layer in the network;  these 
layers allow us to disable certain neurons during training.
 Once we've introducd the basic method, we
 explain how to adapt the training procedure to
support this new layer. %We begin with a high-level overview of our method.

\subsection{Overview}

At a high-level, the ShrinkNets approach consists of two interconnected stages.
The first stage identifies neurons that do not contribute
to increasing the prediction accuracy of the network and deacivate them. The
second stage is concerned with actually removing the neurons from the network
and shrinking its size, therefore leading to faster inference times. We give an
overview of both processes next:

\noindent\textbf{Deactivating Neurons On-The-Fly: }During the first stage,
ShrinkNets applies an on/off switch to every neuron of an initially oversized
network. We model the on/off switches by multiplying each input (or output) of
each layer by a parameter $\bm{\theta}$, with values $0$ or $1$. A value of $0$
will deactivate the neuron, while $1$ will let the signal go through. These
switches are part of a new layer, which we call \textbf{Switch layer} which can
be applied to fully connected as well as convolutional layers.

We want to minimize the number of on switches to reduce the model size as much
as we can, while preserving prediction accuracy. This can be achieved by jointly
minimizing the objective of the network and a factor of the L0 norm of the
vector containing all the on/off switches. Because finding an optimal binary
assignement is an NP-Hard problem, we allow $\bm{\theta}$ to be a real number
instead of a 0/1 value, thus constraining the L1 norm as opposed to the L0 norm.

\noindent\textbf{Neuron Removal: } During this stage, the neurons that are
deactivated by the Switch layers are actually removed from the network,
effectively shrinking the network size. This is what leads to faster inference
times, as we demonstrate in our evaluation. We choose to remove neurons at
training time because we have observed that allows the active neurons to adapt
to the new network architecture. Existing techniques focus on neuron removal
after training, and require an extra fine-tuning process to compensate for the
removal.

In the remainder of this section we describe in detail the Switch layer as well
as how to adapt the training process for ShrinkNets.

%This approach assumes that we start with an upper bound on the model size. This
%obviously translates in a computational overhead. Our insight is that some
%usless neurons (we have multiple definitions below) can be removed early without
%impacting the final solution. It has two practical implications: It mitigate the
%issue we describe but it also allows other neurons to adapt as soon as one of
%their peer is killed. Existing technique usually remove them after convergence
%and require an extra fine-tuning process to compensate for the removal.

%We implement our strategy of dynamically deactivating neurons by means of a 
%specialized neural network layer called the \textbf{Switch Layer}.
%We first describe the key concepts and theory backing the switch layer and then
%provide a brief summary of our implementation of switch layers.

%Our approach is based on the insight that given enough capacity (and time), an 
%optimization routine can find a network that can fit an arbitrary function.
%Therefore, by starting with an oversize network and applying on/off switches to
%each of the neurons in it, we can identify a {\it pruned} version of the network
%that can achieve similar accuracy to the oversized network.
%
%
% If we have an oversized network then there exist a pruned version that can still
% achieve our goal. The main idea is to consider an on/off switch for each neuron,
% and we want to find an assignement for this switches that achieve a certain
% size/accuracy tradeof. 

% The key components of the system are: The filter vectors that simulate the
% continuous on/off switches, a regularization that tries to kill neurons, the
% neuron removal strategy that detects neurons that should probably be removed,
% and the garbage collection that effectively remove dead neurons from the model,
% and the simplifaction procedure that remove the filter vector for fast
% inference.

% We will describe these components in the upcoming sections.

% \subsection{Notations}

% \par In order to avoid any potential ambiguity, in this section we will
% describe in details the mathematical notations used in this article. 
% Non-bold
% letters represent scalar values, while bold lowercase and upper case
% repectively denote vectors and matrices. 
% $\bm{A}^T$ stands for the transpose of
% the matrix $\bm{A}$. 
% Subscripts are used to index particular elements of
% vectors and matrices. 
% $\bm{x}_i$, $\bm{A}_i$, $\left(\bm{A}^T\right)_j$ and
% $\bm{A}_{i,j}$ respectively correspond to the $i^{th}$ component of $\bm{x}$,
% the $i^{th}$ row of $\bm{A}$, the $j^{th}$ column of $\bm{A}$ and the $j^{th}$
% component of the $i^{th}$ row of $\bm{A}$. 
% All the following definitions assume
% $\bm{A}$ to be an $n\times p$ matrix.  
 
% For any $l \in \left[0, +\infty\right]$ we define the norm: $\norm{\bm{A}}_l =
% \left(\sum_{i=1}^n \sum_{j=1}^p \abs{\bm{A}_{i, j}}^l\right)^{\frac{1}{p}}$. 
% For the rest of this paper and unless stated otherwise, $\bm{y}$ will represent the output of a network, $\bm{x}$ the input, $\bm{b}$ a bias, $\lambda$ regularization factors, $\Omega$ regularization methods, $\bm{\theta}$ general model parameters and $a$ will stand for any element-wise activation function. 
% The only constraint that we want to enforce is that $a(0) = 0$. 
% We use $\intint{u, v}$ to denote inteveral of integers, 
% $\bm{0}$ is the null vector (size depending on the context). 
% $\#S$ is meant to represent the cardinality of a set $S$. 
% To simplify the notation of function composition we use the following operator: 
% $g(f(\bm{x})) = (f \circ g)(\bm{x})$ and for a long sequence of functions from 
% $f_1$ to $f_n$ we use: $f_n(...f_1(\bm{x})) = \left(\bigcirc_{k = 1}^n f_k\right)(\bm{x})$.


\subsection{The Switch Layer}

Let $L$ be a layer in a neural network that takes an input tensor $\bm{x}$ and
produces an output tensor $y$ of shape $\left(c \times d_1 \times \dots \times
d_n\right)$ where $c$ is the number of neurons in that layer.  For instance, for
fully connected layers, $n$=0 and the output is single dimensional of size $c$
(ignoring batch size for now) while for a 2-D convolutional layer, $n$=2 and $c$
is the number of output channels or feature maps.

% maps
% For instance, for fully connected layers, $n$=0 and the output is single 
% dimensional of size $c$ (ignoring batch size for now) while for a 2-D 
% convolutional layer, $n$=2 and $c$ is the number of output channels or feature 
% maps.
%  and furthermore, let the output
% of layer $l$ be a tensor of shape $\left(C \times D_1 \times \dots \times 
% D_n\right)$ where $C$ is the number of neurons in $l$.
% In the case of a fully connected layer, $C$ is the dimensionality of the output
% of the layer whereas in the case of a convolutional layer, $C$ is the number of
% feature maps and $\left(D_1 \times \dots \times D_n\right)$ is the size of those feature
% maps.

Suppose we wish to tune the size of $L$ by applying a switch layer.
A switch layer $S$ applied to the output of $L$ can be parametrized by a 
vector $\theta \in \mathbb{R}^c$ such that the result of applying $S$ to $L(\bm{x})$
is a tensor also of size $\left(c \times d_1 \times \dots \times d_n\right)$
such that: 
\begin{equation} 
S_{\theta}(L(\bm{x}))_{i,...} = \bm{\theta}_iL(\bm{x})_{i, ...}
\end{equation}
% where $\diag{\bm{\theta}}$ is a $c\times c$ matrix such that: $\forall
% 1 \leq i \leq c$, $\diag{\bm{\theta}}_{i, i} = \bm{\theta}_i$ and $0$ otherwise. 
Effectively, once passed through the switch layer, each output channel $i$ 
produced by $L$ is scaled by the corresponding $\theta_i$.
If for any $k$, if $\theta_i = 0$, the $i^{\text{th}}$ channel is multiplied by 
zero and wont contribute to any computations after the switch layer.
If this happens, we say the Switch layer has {\it deactivated} or killed the 
neuron of layer $L$ corresponding to channel $i$. 
% Switch layers have weights in the range $[-\infty,+\infty]$ and are usually 
% placed after linear and convolutional layers.
% The \textit{Switch Layer} takes an input of size $\left(B \times C \times D_1
%   \times \dots \times D_n\right)$, where $B$ is the batch size, $C$ the number
% of features (or channels, in the case of convolutional layers), and $D$ any
% additional dimension. 
% This structure makes it compatible with fully connected
% layers with $n=0$ or convolutional layers with $n=2$. 
% Their crucial property is
% a parameter $\theta \in \mathbb{R}^C$. 
% The output is defined as follows: \vspace{-1em}
% \begin{equation} 
% Switch(\bm{I};\bm{\theta}) = \diag{\bm{\theta}} \bm{I}  
% \end{equation}
% where $\diag{\bm{b}}$ a $n\times n$ matrix such that: $\forall
% 1 \leq i \leq n$, $\diag{\bm{b}}_{i, i} = \bm{b}_i$ and $0$ otherwise. 

% Where $\theta$ is expanded in all
% dimensions to match the input size (except the second one since they are equal
% by definition). 

% These disabled neurons/channels can be removed from the network
% without changing its output. 
% Before explaining how that is achieved, we explain
% next how the weights of the Switch Layer are initialized and adjusted during
% training.

\subsection{Training ShrinkNets} 

To tune the size of a network, we place Switch Layers after each layer whose size
we wish to tune; these are typically the fully connected and convolutional layers
in a network.
Since the switch layers are adept at deactivating neurons, we start with an
oversized network (i.e. network with more capacity than required) and then use
switch layers to deactivate or kill off neurons that are unnecessary.

Formally, we can express this procedure in terms of a sparsity constraint that pushes
values in the $\theta$ vector to 0.
Given a neural network parameterized by weights $\bm{W}$ and switch layer 
parameters $\bm{\theta}$, we optimize the following ShrinkNet loss that 
augments the regular training loss with a 
regularization term for the switch parameters and another on the network weights.
\begin{equation}
  L_{SN}(\bm{x},\bm{y};\bm{W}, \bm{\theta}) = L(\bm{x}, \bm{y}; \bm{W}) +
  \lambda\norm{\theta}_1 + \lambda_2\norm{\bm{W}}_p
\end{equation}

%To train networks we need start with a substantially oversized network, then we
%insert \textit{Switch Layers}  (usually after every linear or convolutional
%layer except the last one) and we sample their weight from the
%$\text{Uniform}(0, 1)$ distribution. 
% we could train the network directly using our standard loss function, and we could achieve performance equivalent to a normal neural
% network. However, our goal is to find the smallest network with reasonable
% performance. We achieve that by introducing sparsity in the parameters of the
% \textit{Switch Layers}, thus forcing the deactivation of neurons
%. Indeed, having a negative component in the $\theta$
%parameter of the filter layer permamently disable its associated feature
%\gl{Maybe redundant ? we talked about that in the previous paragraph} . 
% To obtain this sparsity, we simply redefine the loss function:
% \begin{equation}
%   L'(\bm{x},\bm{y};\bm{W}, \bm{\theta}) = L(\bm{x}, \bm{y}; \bm{W}) +
%   \lambda\norm{\theta}_1 + \lambda_2\norm{\bm{W}}_p
% \end{equation}

% The additional term $\lambda|\max(0, \theta)|$ introduces sparsity (see Lasso
% loss~\cite{Tibshirani1996}). 
% The $\lambda$ parameter, that can take any
% positive value, adjusts how aggressively the network deactivates neurons, with
% larger values indicating more aggressive deactivation.
%  The second component of the loss increases the gradient with respect to $\theta$, thus pushing its value towards zero. Neurons with little impact
% on the original loss (gradient lower than $\lambda$), will not be able to
% compete against this attraction towards zero. Because the entries in $\theta$
% with a value of $0$ or less correspond to dead neurons, $\lambda$ effectively
% controls the number of neurons/channels in the entire network. Without the last
% term our problem sounds very similar to the Group Sparsity regularization which
% is well known in the area of linear and logistic regressions. In the next
% section we will try to undercover the relationship between these two problems,
% explain why we need this additional regularization term and what should be the
% value of $p$.

%!TEX root=paper.tex
%\subsection{Relation to Group Sparsity}
\noindent\textbf{Relation to Group Sparsity (LASSO): } ShrinkNets removes neurons,
i.e., inputs and outputs of layers. For a fully connected layer defined as:
%
\begin{equation} \label{fully_connected}
  f_{\bm{A}, \bm{b}}(\bm{x})=a(\bm{Ax + b})
\end{equation}
%
where $\bm{A}$ represents the connections and $\bm{b}$ the bias,
removing an input neuron $j$ is equivalent to having $\left(\bm{A}^T\right)_j =
\bm{0}$. Removing an output neuron $i$ is the same as setting $\bm{A}_i = \bm{0}$
and $\bm{b}_i = 0$. Solving optimization problems while trying to set entire
group of parameters to zero is the goal of group sparsity regularization
\cite{Scardapane2017}. In  any partitioning of the set of parameters $\bm{\theta}$ defining a model in $p$
groups: $\bm{\theta} = \bigcup_{i=1}^P \bm{\theta}_i$, group sparsity is defined as: 
%
\begin{equation}
  \Omega_\lambda^{gp} = \lambda \sum_{i=1}^p \sqrt{\#\bm{\theta_i}} \norm{\bm{\theta_i}}_2 \\
\end{equation}
%
\srm{define notation -- what is $\lambda$;  what is $\#\theta$, what is $\Omega$? Where does the square root come from?} \gl{This is how it is defined in the
original paper, it would require a few sentences to explain that, do really
need to do ?}
In fully-connected layers, the groups are either: columns of
$\bm{A}$ if we want to remove inputs, or rows of $\bm{A}$ and the corresponding
entry in $\bm{b}$ if we want to remove outputs. For simplicity, we focus
our analysis in the simple one-layer case. In this case, filtering outputs does
not make sense, so we only consider removing inputs. The
group sparsity regularization then becomes:
%
\begin{equation} \label{group_sparsity_regularization}
  \Omega_\lambda^{gp} = \lambda \sum_{j=1}^p \norm{\bm{\left(A^T\right)_j}}_2 \\
\end{equation}
%
Because $\forall i, \#\bm{\theta}_i = n$, \ra{is \# the general way of expressing
cardinality? why not $|x|$?,}  \gl{In europe |x| is for the abolute value, and I was using it in the notation section that someone removed, therefore there would have been a notation conflict}to make the notation simpler, we
embedded $\sqrt{n}$ inside $\lambda$.

Group sparsity and ShrinkNets try to achieve the same goal. We discuss next how
they are related to each other. First let's recall the two problems. In the context of approximating $\bm{y}$ with a linear regression from features $\bm{x}$), the
original ShrinkNet problem is 
%
\begin{equation}
  \min_{\bm{A}, \bm{\beta}} \norm{\bm{y} - \bm{A}\diag{\bm{\beta}}\bm{x}}_2^2 + \lambda \norm{\bm{\beta}}_1
\end{equation}
%
And the Group Sparsity problem is:
%
\begin{equation}
  \min_{\bm{A}} \norm{\bm{y} - \bm{A}\bm{x}}_2^2 + \Omega_\lambda^{gp}
\end{equation}
%

We can prove that under the condition: $\forall j\in \intint{1, p},
\norm{\left(\bm{A}^T\right)_j}_2 = 1$ the two problems are equivalent
(\cref{gps_equivalence}, see Apendix). However if we relax this constraint then ShrinkNets
becomes non-convex and has no global minimum
(\cref{unconstrained_non_convex,unconstrained_shrinknet_no_min}, also in Appendix). Fortunately,
by adding an extra term to the ShrinkNet regularization term we can prove that:
%
\begin{equation}
  \min_{\bm{A}, \bm{\beta}} \norm{\bm{y} - \bm{A}\diag{\bm{\beta}}\bm{x}}_2^2 + \Omega_\lambda^s + \lambda_2\norm{A}_p^p
\end{equation}
%
has global minimums for all $p>0$ (\cref{shrinknet_regularized_minimum}).
This is the reason we defined the \textit{regularized ShrinkNet penalty} above
as:
%
\begin{equation}
  \Omega_{\lambda, \lambda_2, p}^{rs} = \lambda\norm{\bm{\beta}}_1 + \lambda_2\norm{\bm\theta}_p^p
\end{equation}
%
In practice we observed that $p=2$ or $p=1$ are good a choice; note that the latter
will also introduce additional sparsity into the parameters because the $L_1$ is, 
thest best convex approximation of the $L_0$ norm.





